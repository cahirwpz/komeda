\documentclass[10pt]{beamer}

\usepackage[utf8]{inputenc}
\usepackage{mathptmx}
\usepackage{amsmath}
\usepackage[normalem]{ulem}
\usepackage{tikz}
\usepackage{hyperref}
\usepackage{listings}
\renewcommand*\ttdefault{txtt}
\usepackage[T1]{fontenc}

\definecolor{links}{HTML}{2A1B81}
\hypersetup{colorlinks,linkcolor=,urlcolor=links}

\lstset{
  basicstyle=\footnotesize\ttfamily,
  numbers=left,
  numbersep=-3em,
  numberstyle=\color{gray}
}

\usetikzlibrary{shapes,arrows,automata}

\tikzset{
  vertex/.style={
    rectangle,
    rounded corners,
    draw=black, thick,
    text centered
  },
}

\mode<beamer>
{
  \usetheme{Frankfurt}
  \useoutertheme{infolines}
  \setbeamercovered{transparent}
}

\subject{Talks}

\AtBeginSection[]
{
  \begin{frame}<beamer>{}
    \tableofcontents[currentsection]
  \end{frame}
}

\title[Komeda]{
  \Huge{Komeda}\\
  \vspace{0.25em}
  \small{Framework for Interactive Algorithmic Music On Embedded Systems}}
\author[Krystian Bacławski]{
  \href{mailto:cahirwpz@cs.uni.wroc.pl}{Krystian Bacławski}, 
  \href{mailto:yagyu@cs.uni.wroc.pl}{Dariusz Jackowski}}
\institute[University of Wrocław]{
  Computer Science Department\\University of Wrocław}
\date{\today}

\begin{document}

\begin{frame}
\titlepage
\end{frame}

\section{Introduction}

\begin{frame}{Introduction}
  \begin{block}{Komeda platform comprises of:}
    \begin{itemize}
      \item music notation language,
      \item hardware independent binary representation,
      \item portable virtual machine capable of music playback,
      \item hardware communication protocol,
      \item interface for sensors and output devices,
      \item support for music generation routines.
    \end{itemize}
  \end{block}
\end{frame}

\section{The language}

\begin{frame}{The design of Komeda language}
  \begin{block}{Design concepts:}
    \begin{itemize}
      \item music split into parts (patterns, phrases, choruses)
      \item rich structure of music score (loops, repetitions, alternatives)
      \item score playback in modified context (transposition, volume change)
      \item virtual players synchronization (transition to next music part)
      \item music interaction with external world (sensors, controllers)
    \end{itemize}
  \end{block}

  \begin{block}{Design challenges:}
    \begin{itemize}
      \item capturing all aforementioned concepts unambiguously,
      \item balancing between having a readable syntax and concise notation,
      \item finding effective mapping to intermediate form (Komeda bytecode),
      \item simplicity of virtual machine and extensions.
    \end{itemize}
  \end{block}
\end{frame}

\begin{frame}[fragile]{Examples (1)}
  \begin{block}{Channel notation:}
    \begin{lstlisting}
      channel 0 play @Pattern0
    \end{lstlisting}
  \end{block}

  \begin{block}{Sample music score:}
    \begin{lstlisting}
      pattern @Example {
        @tempo 120
        @unit 1/4

        C#, D r a3/2 d'/2 e2
      }
    \end{lstlisting}
  \end{block}
\end{frame}

\begin{frame}[fragile]{Examples (2)}
  \begin{block}{Slurrs:}
    \begin{lstlisting}
      pattern @TwoSlurrs {
        a~a/2 c~d~e2~f
      }
    \end{lstlisting}
  \end{block}

  \begin{block}{Pattern invocation:}
    \begin{lstlisting}
      pattern @Pattern0 {
        ...
        @Pattern1
        ... 
      }

      pattern @Pattern1 {
        ...
      }
    \end{lstlisting}
  \end{block}
\end{frame}

\begin{frame}[fragile]{Examples (3)}
  \begin{block}{Temporary parameters modification:}
    \begin{lstlisting}
      pattern @NoteSequence {
        ...
      }

      pattern @FooBar {
        ...
        with @pitch +3 @volume +10% {
          @NoteSequence
        }
        ...
        with @pitch -2 {
          @NoteSequence
        }
        ...
      }
    \end{lstlisting}
  \end{block}
\end{frame}

\begin{frame}[fragile]{Examples (4)}
  \begin{block}{Nested loops and alternatives:}
    \begin{lstlisting}
      pattern @FooBar {
        ...
        times 8 {
          ...
          on 3 {
            ...
            times 2 {
              /* do it twice on 3rd repetition of outer loop */
              ...
            }
          }
          ... 
          on 4 { ... }
        }
        ...
      }
    \end{lstlisting}
  \end{block}
\end{frame}

\begin{frame}[fragile]{Examples (5)}
  \begin{block}{Infinite loop with a break on user's action:}
    \begin{lstlisting}
      @Button = ...

      pattern @FooBar {
        forever {
          ...
          if @Button.pressed()
          break
          ...
        }
      }
    \end{lstlisting}
  \end{block}
\end{frame}

\begin{frame}[fragile]{Examples (6)}
  \begin{block}{Synchronization through meeting points}
    \begin{lstlisting}
      channel 0 play @PatA
      channel 1 play @PatB

      rendezvous @Part2 for 2

      pattern @PatA {
        times 3 { ... }
        ...
        arrive @Part2
        ...
      }

      pattern @PatB {
        ...
        times 2 { ... }
        arrive @Part2
        ...
      }
    \end{lstlisting}
  \end{block}
\end{frame}

\begin{frame}[fragile]{Examples (7)}
  \begin{block}{Communication through signaling}
    \begin{lstlisting}
      signal @StopImprovisation

      pattern @PatC {
        ...
        /* for the drummer: please stop improvising */
        notify @StopImprovisation
        ...
      }

      pattern @PatD {
        ...
        until @StopImprovisation {
          // some crazy beats :)
          ...
        }
        acknowledge @StopImprovisation
        ...
      }
    \end{lstlisting}
  \end{block}
\end{frame}

\begin{frame}[fragile]{Example (8)}
  \begin{block}{Voice definition}
    \begin{lstlisting}
      voice @Drum {
        file "drum.wav"
        pitch a
        sustain 0.2 0.9
      }
    \end{lstlisting}
  \end{block}
\end{frame}

\section{The virtual machine}

\begin{frame}{Binary representation}
  \begin{block}{Reduced number of orthogonal instructions.}
    \begin{description}
      \item[Q:] Could provide $1:1$ mapping between Komeda and native
        instruction set? (portability vs. efficiency)
      \item[A:] {\bf We choose portability!}
    \end{description}
  \end{block}

  \begin{block}{Sizeable register file and a stack.}
    \begin{itemize}
      \item Komeda is about expressing control flow -- not data processing.
      \item No concept of memory (scarce resource).
      \item Virtual registers and stack store program's state.
    \end{itemize}
  \end{block}

  \begin{block}{Uniform size of instructions (2 bytes).}
    \begin{itemize}
      \item Makes it easy to quickly decode and dispatch instructions.
      \item Simplified pattern generation by native code.
      \item Easy decompilation.
    \end{itemize}
  \end{block}
\end{frame}

\begin{frame}{Execution engine}
  \begin{description}
    \item[{\tt NotePlayer}] a scheduler that updates the state of channels and
      reprograms {\tt AudioPlayer} respectively.  For each note being played
      {\tt NotePlayer} maintains an alarm clock.  After a note is expired, next
      one is fetched. Secondary functionality -- the maintenance of
      synchronization state between channels.

    \item[{\tt AudioPlayer}] mixes audio inputs into a stream suitable for
      audio playback device.  Heavily dependent on the platform: {\tt DAC} with
      or without {\tt DMA}, {\tt PWM} generator, {\tt FM} synthesis chip.  {\tt
        AudioPlayer} may perform some computations (e.g. sound synthesis,
      fetching samples from storage device).  Probably the only subsystem
      called periodically by the system clock.

    \item[{\tt Interpreter}] invoked by {\tt NotePlayer} to update the state of
      a channel.  Produces one note or rest at a time.  Executes Komeda
      instructions.

    \item[{\tt NoteGenerator}] optional subsystem written in C language.  Implements
      an algorithm not expressible in Komeda language.  May intercept calls to
      {\tt Interpreter}.
  \end{description}
\end{frame}

\begin{frame}{Modules}
  \begin{block}{Lots of similarities to object oriented paradigm:}
    \begin{itemize}
      \item internal state not visible to Komeda language,
      \item methods (possibly blocking),
      \item properties -- exposed state,
      \item parametrized and non-trivial initialization,
      \item multiple instances per program.
    \end{itemize}
  \end{block}

  \begin{block}{Possible applications:}
    \begin{itemize}
      \item Synthesized voices.
      \item Communication over wireless network.
      \item Sensors and controllers.
    \end{itemize}
  \end{block}
\end{frame}

\section{Future work}

\begin{frame}{We're currently working on...}
  \begin{block}{Implementation for embedded platforms:}
    \begin{enumerate}
      \item Arduino (optionally, with WAVE shield)
      \item 8-bit and 16-bit microcontrollers (PIC, DSPIC, AVR, etc.)
    \end{enumerate}
  \end{block}

  \begin{block}{Establishing interface for plugins:}
    \begin{enumerate}
      \item {\bf Voices}: {\tt DAC} and {\tt FM} chips, {\tt PWM} generators.
      \item {\bf External instruments}: {\tt MIDI} interface, robotic control
        of real instrument.
      \item {\bf Sensors}: light, temp. or movement controlling tempo, volume
        or pitch.
      \item {\bf Generators}: algorithms generating note stream, in {\tt C}
        language.
    \end{enumerate}
  \end{block}
\end{frame}

\begin{frame}{Planned features}
  \begin{enumerate}
    \item Sound synthesis.
    \item Continuous parameter control (linearity, envelopes).
    \item Microtones.
    \item Source code macros.
  \end{enumerate}
\end{frame}

\section*{Questions}

\begin{frame}
  \begin{center}
    \Huge{Questions?}
  \end{center}
\end{frame}

\end{document}
