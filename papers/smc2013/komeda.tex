\documentclass{article}
\usepackage{smacsmc2013}
\usepackage{times}
\usepackage{ifpdf}
\usepackage[english]{babel}
\usepackage{cite}
\usepackage{listings}

\lstset{
  basicstyle=\small\ttfamily,
  captionpos=b
}

%%%%%%%%%%%%%%%%%%%%%%%% Some useful packages %%%%%%%%%%%%%%%%%%%%%%%%%%%%%%%
%%%%%%%%%%%%%%%%%%%%%%%% See related documentation %%%%%%%%%%%%%%%%%%%%%%%%%%
%\usepackage{amsmath} % popular packages from Am. Math. Soc. Please use the 
%\usepackage{amssymb} % related math environments (split, subequation, cases,
%\usepackage{amsfonts}% multline, etc.)
%\usepackage{bm}      % Bold Math package, defines the command \bf{}
%\usepackage{paralist}% extended list environments
%%subfig.sty is the modern replacement for subfigure.sty. However, subfig.sty 
%%requires and automatically loads caption.sty which overrides class handling 
%%of captions. To prevent this problem, preload caption.sty with caption=false 
%\usepackage[caption=false]{caption}
%\usepackage[font=footnotesize]{subfig}

%user defined variables
\def\papertitle{Komeda: Framework for Interactive Algorithmic Music On Embedded
  Systems}
\def\firstauthor{Krystian Bac\l awski}
\def\secondauthor{Dariusz Jackowski}

% pdf-tex settings: detect automatically if run by latex or pdflatex
\newif\ifpdf
\ifx\pdfoutput\relax
\else
   \ifcase\pdfoutput
      \pdffalse
   \else
      \pdftrue
\fi

\ifpdf % compiling with pdflatex
  \usepackage[pdftex,
    pdftitle={\papertitle},
    pdfauthor={\firstauthor, \secondauthor},
    bookmarksnumbered, % use section numbers with bookmarks
    pdfstartview=XYZ % start with zoom=100% instead of full screen; 
                     % especially useful if working with a big screen :-)
   ]{hyperref}
  \pdfcompresslevel=9

  \usepackage[pdftex]{graphicx}
  % declare the path(s) where your graphic files are and their extensions so 
  % you won't have to specify these with every instance of \includegraphics
  \graphicspath{{./figures/}}
  \DeclareGraphicsExtensions{.pdf,.jpeg,.png}

  \usepackage[figure,table]{hypcap}

\else % compiling with latex
  \usepackage[dvips,
    bookmarksnumbered, % use section numbers with bookmarks
    pdfstartview=XYZ % start with zoom=100% instead of full screen
  ]{hyperref}  % hyperrefs are active in the pdf file after conversion

  \usepackage[dvips]{epsfig,graphicx}
  % declare the path(s) where your graphic files are and their extensions so 
  % you won't have to specify these with every instance of \includegraphics
  \graphicspath{{./figures/}}
  \DeclareGraphicsExtensions{.eps}

  \usepackage[figure,table]{hypcap}
\fi

%setup the hyperref package - make the links black without a surrounding frame
\hypersetup{
    colorlinks,%
    citecolor=black,%
    filecolor=black,%
    linkcolor=black,%
    urlcolor=black
}


% Title.
% ------
\title{\papertitle}

% Authors
%--------------
\twoauthors
{\firstauthor}{Institute of Computer Science \\ University of Wroc\l aw\\
  {\tt\href{mailto:krystian.baclawski@ii.uni.wroc.pl}
    {krystian.baclawski@ii.uni.wroc.pl}}}
{\secondauthor}{Institute of Computer Science \\ University of Wroc\l aw \\
  {\tt\href{mailto:dariusz.jackowski@ii.uni.wroc.pl}
    {dariusz.jackowski@ii.uni.wroc.pl}}}

% ***************************************** the document starts here ***************
\begin{document}

\capstartfalse %
\maketitle %
\capstarttrue %

\begin{abstract}
Application of embedded systems to music installations is limited due to
absence of convenient software development tools. This is a very unfortunate
situation as these systems offer a set of advantages in comparison to desktop
or laptop computers. Embedded systems are small in size therefore easier to
incorporate into the form of the work. These devices are effortlessly
expandable with various sensors and controllers. Moreover they are affordable,
which creates possibility to build networks of cooperating devices.

In this paper we describe a design of Komeda -- the platform for interactive
algorithmic music on embedded systems. The framework consists of language based
on the score-with-blanks approach, the intermediate binary representation,
portable virtual machine and module system.
\end{abstract}
%

\section{Introduction}
\label{sec:introduction}
Artists creating music installations  have somewhat limited options for platform running generative software.
Desktop or laptop computers can be to obstructive to the visual and spatial form of the installation.
The mobile solutions (phones, tablets etc.) aren't suited for this application - problems with power and heat, lack of
easy way to incorporate external sensors and output devices makes it somewhat limited for installations
 Thus the most viable choice is
apparently to use small embedded systems, which could be easily built into both
static and movable parts of an installation. Such systems should be based on
affordable components and be expandable with various sensors and output
devices. Another conceivably useful feature is mutual communication between
system components. Hardware that matches the description above already exists,
but it's not backed up by good software framework, that enables easy creation
of musical applications.

The platform should include:
\begin{itemize}
  \item music notation language,
  \item hardware independent binary representation,
  \item portable virtual machine capable of music playback,
  \item hardware communication protocol,
  \item interface for sensors and output devices,
  \item support for music generation routines.
\end{itemize}

We would like to present the Komeda system, which currently only provides
subset of mentioned features, but in the future should encompass all of them. 

\section{Overview} 
\label{overview}

Komeda consists of the following components: language that supports
score-with-blanks approach, intermediate binary representation, virtual machine
and module system. The language is used to create a musical score with place
holders (''blanks'') for the code generated by the desired module. The score
takes form of list of notes organized in patterns which are similar to
parameterless procedures from programming languages. In addition to notes the
language offers some limited set of control structures, patterns invocation and
instructions to communicate with modules. Modules represent instruments,
generator routines and inputs from sensors. Each module is assigned a set of
parameters and actions. The binary representation is platform independent
bytecode for KomedaVM. The virtual machine provides player routine (or more
specifically a scheduler and controller for instrument modules) and interpreter
for the bytecode.

\section{Language}
\label{lang}

\subsection{Design}
\label{lang:design}

As we managed to highlight earlier, Komeda is a complete environment. The first
and the most important component of the platform is the language. It is mainly
focused on providing a succinct music notation description. However, it is not
solely a data definition language like XML since there is a need for some
programming constructs. We enable the user to express easily following
concepts:

\begin{itemize}
  \item music split into parts (i.e. patterns, phrases, choruses),
  \item rich structure of music score (incl. loops, repetitions, alternatives),
  \item playback of score fragment in modified context (i.e.~transposition,
    volume change),
  \item virtual players synchronisation (e.g.~concerted transition to next
    music part), 
  \item music interaction with external world (e.g.~sensors, controllers).
\end{itemize}

Having language users without much of a former experience, we would like them to find the concepts clear and code readable. To help with adoption of Komeda, we decided to base its syntax on something widely known in both worlds of music notation and programming languages. Hence, significant number of concepts were borrowed from LilyPond \cite{lily} and Java. 

The main challenges we experienced while designing Komeda were:
\begin{itemize}
  \item capturing all concepts needed to represent music score unambiguously --
    i.e.~a user should be encouraged to express ideas in the least number of
    possible ways,
  \item balancing between having a readable syntax and concise notation -- note
    that these two aspects collide with each other, as shorter words become
    more cryptic,
  \item finding effective mapping to intermediate form (Komeda bytecode) --
    shape of binary data that represents the music to be executed in VM,
  \item simplicity of virtual machine and extensions (e.g. generators, sensors,
    etc.).
\end{itemize}

Our design went a complete overhaul several times, as we were discovering
dependencies between all three layers of Komeda environment: the language, the
binary representation and the virtual machine.

At the time of writing we have working compiler from Komeda to intermediate
representation. The implementation is written in Haskell, a purely functional
language with strong static typing. Our choice is justified by a few arguments:

\begin{itemize}
  \item Haskell \cite{haskell} has set of rich abstractions and expressive
    types - our code looks clear and simple, there is little verbosity or
    boilerplate code, that obliterates the ideas,
  \item availability of great parser frameworks -- Parsec \cite{parsec} is used
    to express Komeda grammar.
\end{itemize}

\subsection{Theoretical considerations}
\label{lang:theory}

We choose to model the language after western music notation. As it is not
supposed to be music engraving language, we had some flexibility in
implementing musical concepts. In some cases we completely resign from some of
them -- most notable examples are key signatures, bars and time signatures.
Here we would like to supply few arguments for this omissions.

The notion of key signature associated with tonality was made somewhat obsolete
by atonal techniques or scales other than modes of the major system. Moreover
introducing the key signature could result in the decreased readability. Such
is the case with Oliver Messaien's prelude {\it La colombe} notated in E major
key in which most of the chords have more than one accidental, which makes hard
to keep track of exact pitches in signatures with four sharps.  Similarly
notation of music based on the non-standard heptatonic scales requires custom
key signatures (e.g. some of the pieces from Bela Bartok's {\it Microcosmos}
which utilize both flats and sharps in the signature) or usage of accidentals
instead of the key signature.  Scales with more than seven pitch class
appearing for example  in the music of Bartok \cite{bela1} \cite{bela2} and
Messaien (whose modes of limited transposition \cite{mess} mostly have more
than seven tones) cannot be notated without accidentals. The same happens with
works based on the serialism, dodecaphony and intervals pairings \cite{lutos}
or other atonal techniques. There is also another argument for foregoing the
key signature even for tonal works. Contemporary composers often chose to
notate them with only accidentals - great example here is the first movement of
III Symphony by Henryk G\'{o}recki, who always notated F\# with accidental in
this E-minor piece.

Bars and metre are used in music mainly for three things: structuring, as a
help for the performer in keeping track of the score and to introduce default
accents in music. In Komeda the visual structure of music can be imposed using
the white characters and comments. Obviously Komeda does not need any help in
keeping track of the score, so we are left with only last application of
measures. The problem of accent induced by metre is that there is not one
standard of it. Underlying pulse of different time signatures is associated
with genre, period and personal styles of both the composer and the performer.
Moreover in the contemporary music the implicit accent is non-existent
\cite{harm}. The lack of bars also simplify music generation.

\subsection{Syntax concepts}
\label{lang:concepts}

\subsubsection{Channels}
Komeda language is oriented towards describing behavior of monophonic playback
channels. Each channel has an independent thread of execution that interprets
notes and instruction attached with primary music score (pattern).

\begin{lstlisting}[caption=Channel notation example]
channel 0 play @Pattern0
\end{lstlisting}

The definition above tells, that the channel number 0 will play a program
defined by {\tt Pattern0}. Number of channels is limited by the implementation
of KomedaVM.

Note that the program executed by the channel may make use of certain sensor,
voices, generator modules. That information has to be specified in channel
initialization routine that is automatically added by the compiler. Thus code
analysis has to be performed at compile time. 

\subsubsection{Identifiers and Parameters}

Komeda programs will assign names to specific objects like patterns, voices,
etc. Another group of names refer to certain playback parameters like tempo,
volume, etc. Hence two groups of identifiers were conceived to capture these
concepts. All user introduced identifiers begin with “@” sign, followed by
uppercase letter (e.g. {\tt @FooBar}). Parameters follow similar syntax, though
they begin with lowercase letter (e.g. {\tt @tempo}). Some of them are built
into language (i.e. channel control), but others depend on modules imported
into the scope of a channel.

\subsubsection{Patterns}
They are used to give music score a basic structure. Each pattern is given a
unique name and can be referred to from other patterns, as if it was copied
into the place of reference. Pattern execution interprets notes and control
commands hold within its structure.

\begin{lstlisting}[caption=Example of pattern invocation]
pattern @Pattern0 {
  ...
  @Pattern1
  ... 
}

pattern @Pattern1 {
  ...
}
\end{lstlisting}

\subsubsection{Notes, Rests and Slurs}
Very basic concept embraced by Komeda is a note. It is specified by pitch
(semitone) and note length. A close relative of a note is a rest, which stops
playback for a given unit of time. The timing is implicit - i. e. notes starts
when a previous one stops. Pitch is expressed in Helmholtz notation with sharps
only (we chose to resign from flats). Length is represent as a fraction of
default unit. As this is notably more powerful than standard notation we were
able to resign from tuples and dots. Please note that although unit can be set
arbitrary, tempo is expressed always in quarter notes per minute.

\begin{lstlisting}[caption=Sample music score]
@tempo 120
@unit 1/4

C#, D r a3/2 d'/2 e2
\end{lstlisting}

Komeda supports expressing slurs. Also ties are supported, though they`re
immediately coalesced into single note at compilation time.

\begin{lstlisting}[caption=Expressing slurs]
a~a/2 c~d~e2~f
\end{lstlisting}

\subsubsection{Control Structures}
Usually a score consist of a few fragments that are repeated several {\tt
  times}. Certainly one would like to express that concisely, possibly giving
alternative endings (western music notation uses volta brackets for that
purpose). Komeda is more flexible as it allows to designate alternative
fragments (not only endings) {\tt on} specified loop iteration (please note
that alternative fragments don't have to be defined for each iteration, which
can lead to repetitions of different lengths). It's also possible to nest
repetitions as shown below:

\begin{lstlisting}[caption=Nested loops and alternatives]
times 8 {
  ...
  on 3 {
    ...
    times 2 {
      /* do it twice on 3rd repetition
         of outer loop */
      ...
    }
  }
  ... 
  on 4 { ... }
}
\end{lstlisting}

However sometimes one would like to express possibly infinite repetition that
can be terminated under certain condition (e.g. user interaction). That is
achievable using {\tt forever} construct.

\begin{lstlisting}[caption=Infinite loop with break on user action]
forever {
  ...
  if @Button.pressed()
    break
  ...
}
\end{lstlisting}

\subsubsection{Context Changes}

There are several cases, when a user would like to perform series of similar
actions, captured by a pattern, but with slightly different settings. Example
of such situation is when there is a pattern representing some phrase which
should be repeated with different transposition.

\begin{lstlisting}[caption=Temporary parameters modification]
with @pitch +3 @volume +10% {
  @NoteSequence
}

with @pitch -2 {
  @NoteSequence
}
\end{lstlisting}

The with statement can be used to temporarily change a set of parameters for
the commands placed in its scope.

\subsubsection{Synchronisation}

Because channels are inherently independent, program executing in one channel
has no way to obtain information about state of another channel. While this
isolation is generally a good idea, it poses a problem when timing between
channels is needed to achieve desired effect (e.g. smooth transition from one
music part to another). To synchronize players, we introduced simple, but
effective mechanisms based on the idea of notifications and rendezvous points.

In example below we set a meeting point for two players. They will only advance
to the next part of music, when they both arrive at {\tt @Part2}.

\begin{lstlisting}[caption=Use of meeting points for synchronization]
rendezvous @Part2 for 2

pattern @PatA {
  times 3 { ... }
  ...
  arrive @Part2
  ...
}

pattern @PatB {
  ...
  times 2 { ... }
  arrive @Part2
  ...
}
\end{lstlisting}

Another example requiring synchronization is two players, who improvise at the
beginning and then want to transition to main theme of work.

\begin{lstlisting}[caption=Communication through signalling]
signal @StopImprovisation

pattern @PatC {
  ...
  /* for the drummer:
     please stop improvising */
  notify @StopImprovisation
  ...
}

pattern @PatD {
  ...
  until @StopImprovisation {
    // some crazy beats :)
  }
  ...
}
\end{lstlisting}

\subsubsection{Instruments}

Currently only PCM instruments are supported. Note that at the time of writing
the compiler supports only WAVE format. Optionally a base pitch and beginning
and end of looped part of the sample can be given.

\begin{lstlisting}[caption=Example voice definition]
voice @Drum {
  file "drum.wav"
  pitch a
  sustain 0.2 0.9
}
\end{lstlisting}

\section{Virtual Machine}

KomedaVM is a small piece of software residing in flash memory of an embedded
device. It is responsible for loading (from flash, RAM or if applicable some
external sources such as SD cards) binary representation of Komeda language and
running it. For this the machine has to maintain global state and private state
of each channel. Additionally it manages modules in use and schedule execution
of instruments playback. Certain assumptions described in this chapter
influenced shape of the language or imposed particular constraints on the
design (e.g. number of available channels).  

\subsection{Binary representation}

Almost all Komeda language concepts have their counterparts in binary
representation. Some of composite high-level instructions are split down into
simpler constructs as the compiler lowers the code. At the very bottom each
channel behaves as an independent state machine - a very simple microprocessor
with specialized set of instructions. We will not provide details of
instruction set architecture (ISA) employed by KomedaVM, as it is still
undergoing significant changes.

Having spent considerable amount of time analyzing contemporary
micro-controllers ISAs, we decided to keep close to these designs. Affinities we
would like to preserve are:

\begin{description}
  \item[Reduced number of orthogonal instructions.]If we keep virtual instruction
    design close enough to existing ISA, for instance AVR \cite{avr}, we possibly
    could provide one-to-one mapping between Komeda and native instructions. That
    could reduce the size of interpreter as the processor is able to execute some
    instructions directly. While it is tempting to follow such path, certainly
    specialization hinders portability of virtual machine, which at the moment is
    our priority goal. Thus, we decide to take opposite approach, and deliver
    instructions that can be interpreted easily on most popular embedded
    processors, which tend to have RISC-like ISAs \cite{risc}.

  \item[Sizeable register file and a stack.] Komeda language is mainly oriented
    towards expressing control flow - it is not suitable for data processing.
    Thus, we decided to drop concept of program is memory, which anyway is a
    scarce resource in embedded systems. Instead, we use virtual registers and
    stack to store state of a program. The requirement for virtual stack emerged
    as we considered situation, when a pattern invokes another pattern - in some
    sense it mimics concept of function calls and activation records from
    general purpose programming languages.

  \item[Uniform size of instructions i.e. 2 bytes.] If Komeda binary
    representation is going to be interpreted in software, the interpreter must
    be able to quickly decode and dispatch instructions. Such representation is
    also convenient, if we allow any arbitrary Komeda code processing (e.g.
    decompilation or whole pattern generation by native code).
\end{description}

\subsection{Modules}

Modules from the virtual machine point of view are  independent entities. The
communications with them is performed via specialized modules address space,
which has to be initialized by KomedaVM before execution of a program. The
machine could also call a specific action of a module which virtually means
call of the native code (i.e. not part of KomedaVM). This is the only possible
situation in which control is transferred from Komeda code to the external
native code.

Lets consider a single module. Firstly, we would like to avoid exposing the
state of a module to the virtual machine, if possible. Secondly, when you
consider module's internal state, it may be too big to represent by the
registers visible to KomedaVM. Thus the machine has to maintain the state that
is not visible to Komeda code directly. Each module may be subjected to
non-trivial initialization. Finally, a program may use same module but
differently initialized or used in two different independent contexts.

As a consequence, we decided to design modules in spirit of object oriented
paradigm, but without inheritance. Lots of similarities emerged - clearly
modules are classes, modules with attached state are class instantiations
(i.e.~objects), parameters are public properties of an object, and remote
procedures - just methods. Such model can be efficiently mapped onto C language
which lacks of OO features.

\subsection{Execution engine}

Runtime system is composed of three components characterized shortly below.

\begin{description}
  \item[NotePlayer] Central subsystem of Komeda runtime. It is a scheduler that
    takes care of updating the state of channels (i.e. note pitch and length,
    slur mode, instrument number) and reprogramming AudioPlayer respectively.
    For each note being played NotePlayer maintains an alarm clock, which is
    triggered when the note is about to stop being played. Next note is then
    fetched and AudioPlayer reprogrammed to play it. Secondary functionality is
    related to the maintenance of synchronization state between channels.

  \item[AudioPlayer] Its main task is to continuously mix all audio inputs into
    a stream suitable to be digested by audio playback device. This subsystem
    is considered to be the most platform dependant constituent. Playback
    device implementation may vary greatly for each hardware platform. It can
    be implemented as a DAC with or without DMA (i.e. DAC capable of feeding
    itself with consecutive samples without involving processor), a simple PWM
    generator, FM synthesis sound chip. AudioPlayer may be programmed to
    perform certain non-trivial computation like sound synthesis, or system
    interaction like fetching samples from storage device. It seems to be the
    only subsystem that needs to be called synchronously by the system clock.

  \item[Interpreter] Subsystem invoked by NotePlayer to update the state of a
    channel. Each interpreter invocation has to end up producing a note
    information or a rest, eventually forcing interpreter to enter wait state
    on that channel. The result is obtained by executing compiled Komeda
    language statements.

  \item[NoteGenerator] An optional subsystem written in C language that either
    was assigned to a channel by a programer or temporarily suppressed Komeda
    interpreter and intercepted its execution. It employs certain algorithm to
    deliver notes or rests upon channel update action.
\end{description}

\section{Future Work}

\subsection{Features currently in development}

At the time of writing Komeda is under active development. That means some of
its interesting features undergo implementation process and are being evaluated
with regards to language and virtual machine design. Our long term focus is the
support for embedded platforms, such as Arduino (with optional WAVE shield),
8-bit and 16-bit microcontrollers (PIC, DSPIC, AVR, etc.) with built-in or
attached Digital-to-Analog converters.

One of currently identified design issues is platform specific instruments
support. For many embedded devices, especially those without D/A converter,
implementation of DDS-based instruments is either cumbersome (complex sound encoding)
or utterly impractical (overhead of DDS emulation too high). On the other hand,
those devices are likely to have some native support for generation of simple
waves, e.g. PWM-based square wave generator. Other conceivable platform
specific instruments are external devices controlled via MIDI interface or
mechanical devices that control real instruments. Whole range of possibilities
is available here - from simple sine wave generator, through MIDI synthesizers,
to another embedded device controlling servo arm playing xylophone.

Extending Komeda with non-DDS instruments is not trivial and requires VM
plug-in system. The core of the virtual machine is going to offer the same
functionality as current version excluding PCM instruments. 

Basically same programming interface can be extended to modularize the system
to greater extent.  Two more types of pluggable modules save instruments are
envisioned: sensors and generators.

\begin{itemize}
  \item Sensors purpose is to measure physical quantities such as light
    intensity, magnetic field direction, temperature, etc.; and interpret them
    in a context of music. The values coming from sensors could be used to play
    a note, set new tempo rate or voice volume, i.e. control virtually any
    other parameter available. Example mapping could translate room temperature
    into music tempo (i.e. hotter/cooler into faster/slower) or light intensity
    into base pitch of notes being played (i.e. lighter / darker into higher /
    lower).

  \item Generators are additional mechanism for employing generative music
    techniques. Their role is to deliver a number of values upon request - a
    user has freedom to use these values as she wishes. Alternatively the
    channel could be set to execute code generated by the module until the
    generator returns control. At the moment of writing music generation can
    only be achieved by external code modifying Komeda pattern binary
    representation. While this method is the most powerful one, almost
    certainly is the most difficult to use properly. 
\end{itemize}

\subsection{Planned features}

Arguably the most needed feature of Komeda is language supported sound
synthesis (instead of just instruments as external modules). There are few
issues worth mentioning:

\begin{itemize}
  \item There is a need to extend the language to easily supply required
    synthesis arguments in a concise way.

  \item In chosen model the notes are the only acoustic events. They are
    mutually independent, save slurs, and take parameters only at
    initialization time. Though suitable for representing musical score this
    approach is somewhat limited. Especially when it comes to representing
    sounds with continuously changing spectra or indefinite pitch, etc. 	

  \item Last but not least, if Komeda was to support synthesis techniques, it
    would be necessary to extend the language with instrument specification.
    There are two possibilities to incorporate this into the main language.
    First option is to extend voice definitions with a new language for
    instrument specification. Such language could be modelled after CSound
    orchestra \cite{csound} or Nyquist {\tt .alg} \cite{alg} files. Second
    option is to extend Komeda language with extra constructs expressing
    additional sonic events inside patterns. The later approach would greatly
    increase the expressive power of the language, but could lead to
    performance loss and increased complexity of the system. 
\end{itemize}

Another important deficiency of Komeda is absence of continuous parameter
control. That makes it impossible to express portamento and similar effect.
Create ritardando or accelerando and crescendo or decrescendo is very
problematic, as it needs assignment of different tempo or volume to each
consecutive note. We find three possibilities to solve that -- to allow only
linear function, piecewise linear (i.e. envelopes) or arbitrary functions (i.e.
implemented as arrays).

Another desired addition is source code macros. They would be useful for
implementing of some non essential features. We mean symbolic description of
tempo or volume levels, flats, mordents and other ornaments. In addition, given
enough expressive power, the macros could be used to create tools for
generating or transforming score according to some compositional system. We are
considering both text-substitution and compile time macros.

Another feature which would increase expressive power is introduction of
microtones and tunings other than equal temperament. The later is easier, as we
represent pitch by discrete number.

\section{Conclusions}

Komeda is a complex system and describing its intricacies goes way beyond the
scope of this paper. Instead we decided to present its philosophy and design
decisions. We also chose to list possibilities encountered during development
stages. We hope that people creating similar platforms will find these
information useful.

As for the future of Komeda, we hope to release a first public version in the
next few months. As more and more planned features are being incorporated, the
platform calls for a full evaluation in form of installation or novel
instrument. We are going to choose several artists for cooperation. Moreover,
we would like reiterate some concepts from the Komeda, to provide even more
flexible and easy to use platform.

%%%%%%%%%%%%%%%%%%%%%%%%%%%%%%%%%%%%%%%%%%%%%%%%%%%%%%%%%%%%%%%%%%%%%%%%%%%%%
%bibliography here
\bibliography{komeda}

\end{document}
