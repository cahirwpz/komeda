\documentclass[final,hyperref={pdfpagelabels=false}]{beamer}
\mode<presentation>
  {
    %\usetheme{Berlin}
   \usetheme{Wroc}
  }
  \usepackage{times}
  \usepackage{amsmath,amsthm, amssymb, latexsym}
  \boldmath
  \usepackage[english]{babel}
\usepackage[T1]{fontenc} 
  \usepackage[latin1]{inputenc}
  \usepackage[orientation=portrait,size=a0,scale=1.4,debug]{beamerposter}

  %%%%%%%%%%%%%%%%%%%%%%%%%%%%%%%%%%%%%%%%%%%%%%%%%%%%%%%%%%%%%%%%%%%%%%%%%%%%%%%%%5
  \graphicspath{{figures/}}
  \title[Komeda]{Komeda: Framework for Interactive Algorithmic Music On Embedded
  Systems}
  \author[Baclawski & Jackowski]{Krystian Bac\l awski, Dariusz Jackowski}
  \institute[University of Wroclaw]{University of Wroc\l aw\\ Institute of the Computer Science, Wroc\l aw, Poland}



  %%%%%%%%%%%%%%%%%%%%%%%%%%%%%%%%%%%%%%%%%%%%%%%%%%%%%%%%%%%%%%%%%%%%%%%%%%%%%%%%%5
  \begin{document}
  \begin{frame}{} 

    \begin{block}{\large Overview}
	\begin{itemize}
	\item New framework for creating algorithmic music on minimalistic embedded systems.
	\item It consists of music description language, virtual machine and modules systems. 
	\end{itemize}
    \end{block}
    \begin{columns}[t]
      \begin{column}{.49\linewidth}
        \begin{block}{The Komeda Language}
         
		\end{block}
	        \begin{block}{Example}
	         
        \end{block}
        
      \end{column}
      \begin{column}{.49\linewidth}
        \begin{block}{Architecture}
         
        \end{block}
        \begin{block}{Modules}

        \end{block}

      \end{column}
    \end{columns}

 

    \begin{block}{\large Future Work}


    \end{block}

  \end{frame}
\end{document}


%%%%%%%%%%%%%%%%%%%%%%%%%%%%%%%%%%%%%%%%%%%%%%%%%%%%%%%%%%%%%%%%%%%%%%%%%%%%%%%%%%%%%%%%%%%%%%%%%%%%
%%% Local Variables: 
%%% mode: latex
%%% TeX-PDF-mode: t
%%% End:
